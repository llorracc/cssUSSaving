
We use this simple, partial-equilibrium framework (instead of a richer but less tractable HA-GE model) because our goal is to estimate a unified structural saving model whose ``deep'' parameters are jointly identified using both business-cycle-frequency fluctuations AND long-term trends.  Both cyclical and secular changes in the saving rate have been large, and a model that matched one set of facts (secular, or cyclical) but had strongly counterfactual implications for the other would not be a satisfying description of history.

We are not aware of prior papers that attempt this.  Essentially all of the ``structural'' literature has focused on cyclical dynamics, using one of two frameworks:
\begin{enumerate}
\item Complex Heterogeneous Agent New Keynesian models (HANK) with serious treatments of uncertainty
\item Simple spender--saver Two-Agent New Keynesian models (TANK)
\end{enumerate}

\hypertarget{HA-Model-Too-Hard-Right-Now}{}
\hypertarget{HA-Models-Not-Used-Yet-For-Forecasting}{}
For the first class of models, the last decade or so has seen impressive progress in the degree of realism achievable in the treatment of uncertainty, liquidity constraints, household structure, and other first-order elements of households' microeconomic environment. A flourishing literature today explores the implications of these complexities for many questions; see \cite{kmpHandbook} for a survey. However, at the current state of the art, estimating such a model would be a considerable challenge, as estimation is generally several orders of magnitude more computationally demanding than simulating a calibrated version (which is what the existing literature has done).  Furthermore, rich microfounded models have usually been optimized for the specialized purposes of examining in great depth a single question (such as the mechanisms of transmission of monetary policy, \cite{kmvHANK}) rather than, as we are attempting, to address a number of different causal mechanisms, over different time horizons, simultaneously.  Finally, even if such a rich HA-GE model could be estimated, it is not clear that the extra microeconomic realism would be worth the cost in transparency and tractability.

Indeed, in advocating the use of TANK models, \cite{dgTANK} argue that the simplicity and transparency of models with only two agents more than make up for their lack of fidelity to microeconomic facts.  This point is reflected in the fact that, at present, central banks and other entities that require a workhorse model for current analysis, have (to our knowledge) not gone further than TANK models in their incorporation of heterogeneity. Nonetheless, a major drawback of the TANK models is that they allow no role for uncertainty either as an impulse or a propagation mechanism, despite the large literature in recent years that has argued the uncertainty is a core element of business cycle dynamics (see, e.g., cf.\ \cite{bfjstUncertain}).

Our  model occupies a middle ground. We have only a single agent (making our model simpler in one respect even than the TANK models), but that agent's consumption function is nonlinear (making it harder to analyze analytically).  This nonlinearity brings a major benefit, though, in providing a way to accommodate two mechanisms that do not meaningfully exist in TANK models: Uncertainty and credit constraints.  Furthermore, like a TANK model, its parameters can be straightforwardly estimated to match targeted macroeconomic facts (in our case, the saving rate).

\hypertarget{Our-Model-Has-Been-Used-In-Other-Countries}{}
Likely because of its tractability and simplicity, our model (as introduced in a draft version of this paper) has been found useful as a tool to understand saving dynamics in a number of countries in addition to the US. The model has been used explicitly to forecast consumption and the saving rate at the Bank of England (\cite{BoE_forecasting}).  \cite{modyEtAl_precSaving} use a version of it to motivate an empirical exercise which concludes that labor income uncertainty contributed by at least two fifths to the increase in the saving rate in advanced economies during the Great Recession (consistent with our structural estimates below). \cite{Trichet_JacksonHoleSpeech} argues (referring to our model) that the precautionary motive contributed to the high saving rate in advanced economies after 2008.

\hypertarget{Why-We-Do-Not-Endogenize-Asset-Prices}{}

\subsubsection{Why We Do Not Endogenize Asset Prices}{}

Arguably a deeper problem, both with our paper and with the other literature cited above, is the choice to take as exogenous some of the variation that we would most urgently like to understand.  In particular, our model's finding that a `wealth effect' explains part of the increase in saving in the Great Recession begs the question of what caused the asset price movements that underlie the wealth effect (in stocks, housing and bonds). If, as seems likely, an important driver of asset prices is the degree of uncertainty (cf.\ \cite{bexUncertaintyAssetPrices}, \cite{drechslerUncertainty}), then our method will substantially underestimate the cyclical importance of uncertainty, attributing part of uncertainty's true effect to developments (asset prices; credit availability) that are themselves consequences of uncertainty.

A vast literature has attempted to model asset pricing in general equilibrium.  While some progress has been made in understanding the cross-sectional heterogeneity of asset holdings (cf.\ \cite{gmAssetPricing}), for our purposes what is needed is a model that can capture the cyclical and secular time series of returns.  The extent to which no consensus exists is highlighted by the diversity of the recent literature that has sought to endogenize the precipitous decline of net worth and house prices during the Great Recession.  In this attempt, different authors have built into their models a number of alternative mechanisms, including the presence of a exogenous but rare Great-Depression-like state, exogenous shocks to expectations, or endogenous changes in illiquidity of housing.\footnote{In more detail, building on the literature on consumption disaster risk, \cite{glover:intergenRedistr} adopted a setup in which the aggregate shock includes a Great-Depression-like state. Related work attempted to capture the dynamics of house prices during the last boom and (deep) bust. For example, \cite{kmv:houseBoomBust} show that changes in beliefs about future housing demand can match the volatile dynamics of house prices and house price--rent ratios; but invoking unobservable changes in opinions about the future demand for assets is only a small step from explicitly assuming that asset prices are exogenous. \cite{garrigaHedlund} argue that an endogenous decline in housing liquidity (induced by directed search to buy houses) amplifies recessions by contracting credit and depressing consumption.  The debate on the role of beliefs about house prices, changes in credit supply or mortgage market arrangements includes important contributions of \cite{favilukis:housing}, \cite{kmv:houseBoomBust}, \cite{justPrimTamb:CredSupplyAndHouseBoom}, and \cite{garrigaHedlund}. \label{foot_housingLit}}  The existence of this literature suggests that no single model of asset pricing is adequate both for ``normal'' times and for the Great Recession; more broadly, it seems fair to say that no single asset pricing model has come to be viewed as robustly applicable to most times and places, or for both high-frequency cyclical and low-frequency secular movements in asset prices.  If we were to incorporate any non-consensus model of asset pricing (and, at this point, \textit{all} asset-pricing models are non-consensus models), our paper would inevitably (and correctly) judged to be more about the performance of that asset pricing model than anything else.\footnote{Essentially the same points could be made about our choice to take credit supply as exogenous.  Again, to the extent that movements in credit supply are caused by movements in uncertainty, our estimates may seriously underestimate the contribution of uncertainty to business cycle fluctuations.}

% Our aim is to analyze at various time frequencies the drivers of the time-series dynamics of the aggregate saving rate, a variable that is key for thinking about questions, such as how fast an economy recovers from a recession. We believe for this purpose it is important to \emph{estimate} the model (as opposed to calibrating it); as a result,  we chose a much simpler setup than the literature we just described. (On the other hand, the simple model we present would do a bad job at fitting cross-sectional distributions and would not be useful for, e.g., modelling distributional implications of changes in asset prices.)

The exogeneity assumptions bring us to a final reason for using our tractable model, which is that a central purpose of this paper has been to bring to light the existence of some surprisingly simple empirical relationships between the saving rate and our three explanatory variables.  The construction of an elaborate model that required many pages to set up and explain, and many more pages to estimate, might have drawn attention away from the simplicity of the empirical foundations of the paper, in which the key results are evident even in the OLS reduced form estimates.  Our penultimate section \ref{sReducedFormRegressions} examines the empirical performance of our model in comparison with a number of alternatives (including the reduced form model) and argues that our structural model has advantages over any of them.
